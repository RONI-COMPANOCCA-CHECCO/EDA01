%package list
\documentclass{article}
\usepackage[top=3cm, bottom=3cm, outer=3cm, inner=3cm]{geometry}
\usepackage{multicol}
\usepackage{graphicx}
\usepackage{url}
%\usepackage{cite}
\usepackage{hyperref}
\usepackage{array}
%\usepackage{multicol}
\newcolumntype{x}[1]{>{\centering\arraybackslash\hspace{0pt}}p{#1}}
\usepackage{natbib}
\usepackage{pdfpages}
\usepackage{multirow}
\usepackage[normalem]{ulem}
\useunder{\uline}{\ul}{}
\usepackage{svg}
\usepackage{xcolor}
\usepackage{listings}
\lstdefinestyle{ascii-tree}{
    literate={├}{|}1 {─}{--}1 {└}{+}1 
  }
\lstset{basicstyle=\ttfamily,
  showstringspaces=false,
  commentstyle=\color{red},
  keywordstyle=\color{blue}
}
%\usepackage{booktabs}
\usepackage{caption}
\usepackage{subcaption}
\usepackage{float}
\usepackage{array}

\newcolumntype{M}[1]{>{\centering\arraybackslash}m{#1}}
\newcolumntype{N}{@{}m{0pt}@{}}


%%%%%%%%%%%%%%%%%%%%%%%%%%%%%%%%%%%%%%%%%%%%%%%%%%%%%%%%%%%%%%%%%%%%%%%%%%%%
%%%%%%%%%%%%%%%%%%%%%%%%%%%%%%%%%%%%%%%%%%%%%%%%%%%%%%%%%%%%%%%%%%%%%%%%%%%%
\newcommand{\itemEmail}{rcompanocca@unsa.edu.pe}
\newcommand{\itemStudent}{Roni Companocca Checco}
\newcommand{\itemCourse}{}
\newcommand{\itemCourseCode}{}
\newcommand{\itemSemester}{III}
\newcommand{\itemUniversity}{Universidad Nacional de San Agustín de Arequipa}
\newcommand{\itemFaculty}{Facultad de Ingeniería de Producción y Servicios}
\newcommand{\itemDepartment}{Departamento Académico de Ingeniería de Sistemas e Informática}
\newcommand{\itemSchool}{Escuela Profesional de Ingeniería de Sistemas}
\newcommand{\itemAcademic}{2024 - A}
\newcommand{\itemInput}{Del 01 de Abril 2024}
\newcommand{\itemOutput}{Al 05 Mayo 2024}
\newcommand{\itemPracticeNumber}{01}
\newcommand{\itemTheme}{Revisión de POO y Git & GitHub}
%%%%%%%%%%%%%%%%%%%%%%%%%%%%%%%%%%%%%%%%%%%%%%%%%%%%%%%%%%%%%%%%%%%%%%%%%%%%
%%%%%%%%%%%%%%%%%%%%%%%%%%%%%%%%%%%%%%%%%%%%%%%%%%%%%%%%%%%%%%%%%%%%%%%%%%%%

\usepackage[english,spanish]{babel}
\usepackage[utf8]{inputenc}
\AtBeginDocument{\selectlanguage{spanish}}
\renewcommand{\figurename}{Figura}
\renewcommand{\refname}{Referencias}
\renewcommand{\tablename}{Tabla} %esto no funciona cuando se usa babel
\AtBeginDocument{%
	\renewcommand\tablename{Tabla}
}

\usepackage{fancyhdr}
\pagestyle{fancy}
\fancyhf{}
\setlength{\headheight}{30pt}
\renewcommand{\headrulewidth}{1pt}
\renewcommand{\footrulewidth}{1pt}
\fancyhead[L]{\raisebox{-0.2\height}{\includegraphics[width=3cm]{logo_episunsa.png}}}
\fancyhead[C]{\fontsize{7}{7}\selectfont	\itemUniversity \\ \itemFaculty \\ \itemDepartment \\ \itemSchool \\ \textbf{\itemCourse}}
\fancyhead[R]{\raisebox{-0.2\height}{\includegraphics[width=1.2cm]{abet.png}}}
\fancyfoot[L]{sistema para ingresar datos de alumnos universitarios.}
\fancyfoot[C]{\itemCourse}
\fancyfoot[R]{Página \thepage}

% para el codigo fuente
\usepackage{listings}
\usepackage{color, colortbl}
\definecolor{dkgreen}{rgb}{0,0.6,0}
\definecolor{gray}{rgb}{0.5,0.5,0.5}
\definecolor{mauve}{rgb}{0.58,0,0.82}
\definecolor{codebackground}{rgb}{0.95, 0.95, 0.92}
\definecolor{tablebackground}{rgb}{0.8, 0, 0}

\lstset{frame=tb,
	language=bash,
	aboveskip=3mm,
	belowskip=3mm,
	showstringspaces=false,
	columns=flexible,
	basicstyle={\small\ttfamily},
	numbers=none,
	numberstyle=\tiny\color{gray},
	keywordstyle=\color{blue},
	commentstyle=\color{dkgreen},
	stringstyle=\color{mauve},
	breaklines=true,
	breakatwhitespace=true,
	tabsize=3,
	backgroundcolor= \color{codebackground},
}

\begin{document}
	
	\vspace*{10px}
	
	\begin{center}	
		\fontsize{17}{17} \textbf{ Informe de Laboratorio \itemPracticeNumber}
	\end{center}
	\centerline{\textbf{\Large Tema: \itemTheme}}
	%\vspace*{0.5cm}	

	\begin{flushright}
		\begin{tabular}{|M{2.5cm}|N|}
			\hline 
			\rowcolor{tablebackground}
			\color{white} \textbf{Nota}  \\
			\hline 
			     \\[30pt]
			\hline 			
		\end{tabular}
	\end{flushright}	

	\begin{table}[H]
		\begin{tabular}{|x{4.7cm}|x{4.8cm}|x{4.8cm}|}
			\hline 
			\rowcolor{tablebackground}
			\color{white} \textbf{Integrantes} & \color{white}\textbf{Escuela}  & \color{white}\textbf{Asignatura}   \\
			\hline 
			{\itemStudent \par \itemEmail} & \itemSchool & {\itemCourse \par Semestre: \itemSemester \par Código: \itemCourseCode}     \\
			\hline 			
		\end{tabular}
	\end{table}		
	
	\begin{table}[H]
		\begin{tabular}{|x{4.7cm}|x{4.8cm}|x{4.8cm}|}
			\hline 
			\rowcolor{tablebackground}
			\color{white}\textbf{Practica} & \color{white}\textbf{Tema}  & \color{white}\textbf{Duración}   \\
			\hline 
			\itemPracticeNumber & \itemTheme & 02 horas   \\
			\hline 
		\end{tabular}
	\end{table}
	
	\begin{table}[H]
		\begin{tabular}{|x{4.7cm}|x{4.8cm}|x{4.8cm}|}
			\hline 
			\rowcolor{tablebackground}
			\color{white}\textbf{Semestre académico} & \color{white}\textbf{Fecha de inicio}  & \color{white}\textbf{Fecha de entrega}   \\
			\hline 
			\itemAcademic & \itemInput &  \itemOutput  \\
			\hline 
		\end{tabular}
	\end{table}

 \section{URL DE REPOSITORIO GITHUB}
	\begin{itemize}
		\item URL para el Repositorio GitHub.
		\item \url{https://github.com/RONI-COMPANOCCA-CHECCO}
		\item URL para el laboratorio 1 en el Repositorio GitHub.	
        \item \url{https://github.com/RONI-COMPANOCCA-CHECCO/EDA01}
	\end{itemize}
 
    \section{TAREA}
	\begin{itemize}
		\item Cree una cuenta de usuario en GitHub usando su correo institucional.
		\item opcional por ahora Configure su cuenta de estudiante (https://education.github.com/pack).
		\item Cree un nuevo proyecto personal y desarrolle el ejercicio resuelto en clase. Haga 3 commits como mínimo y muéstrelos. Commit para "¡Hola mundo!", otro para "Bienvenida al curso" y otro para imprimir su nombre.
        \item Cree un proyecto grupal para trabajo colaborativo (de 3 a 5 integrantes).
        \item Cree un archivo por cada tema del manual de java (https://www.w3schools.com/java/default.asp), haga commit e inluyalo en su informe grupal (Divídanse los temas).
 Java Tutorial
 Java Methods
\item Cree ramas para cada integrante y cada cierto tiempo una las ramas al main. No elimine nada para
evidenciar ramas, main y commits.
	\end{itemize}

        \subsection{usuario en GitHub usando su correo institucional.}
 
             \includegraphics[height=9cm]{git.jpeg}
        \subsection{Mi primer push de README.md}
 
             \includegraphics[height=9cm]{push.jpeg}
            \subsection{El commit para Hola mundo}
 
             \includegraphics[height=9cm]{1.jpeg}
             \subsection{El commit para Bienvenido al curso}
 
             \includegraphics[height=9cm]{2.jpeg}
             \subsection{El commit para imprimir el nombre}
 
             \includegraphics[height=9cm]{3.jpeg}

    \section{CUESTIONARIO}
	\begin{itemize}
		\item ¿Por qué Git y GitHub son herramientas importantes para el curso?
        \\
        \text La primera para mi seria la colaboracion que se puede tener con los compañeros para un trabajo en equipo, tambien esta que nos sirve como portafolio pra nuestros trabajos.
		\item ¿Qué conductas éticas deberían promocionarse cuando se usa un Sistema de Control de versiones?
        \\
        \text Transparencia ya que los cambios en el código deben ser claros y coherentes para que se comprenda de manera sencilla tambien esta el respeto a la privacidad, los desarrolladores se deben respetar la privacidad de los demás. No deben intentar obtener acceso a información privada o confidencial a menos que sea absolutamente necesario para el proyecto3.
		\item ¿Qué son los estándares de codificación?
        \\
        \text Segun nuestro entendimiento los estándares de codificación son un conjunto de principios, pautas y reglas que guían a los desarrolladores de software sobre cómo estructurar su código

    \section{Conclusiones}
	\begin{itemize}
		\item La conclusión que yo podria llegar seria que, GitHub es esencial porque nos facilita la colaboración en proyectos de software, ademas nos proporciona un sistema de control de versiones, asi como tambien nos permite compartir código de manera abierta y gratuita, y fomenta la interacción y el intercambio dentro de la comunidad de desarrolladores. por eso deveria ser una herramienta indispensable para nosotros.
	\end{itemize}
	
	\section{REFERENCIAS}
	\begin{itemize}
		\item \url{https://www.hostinger.es/tutoriales/que-es-github}
		\item \url{https://www.techtarget.com/searchitoperations/definition/GitHub}
	\end{itemize}
	
%\clearpage
%\bibliographystyle{apalike}
%\bibliographystyle{IEEEtranN}
%\bibliography{bibliography}
			
\end{document}

